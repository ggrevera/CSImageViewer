\section{pnm\-Helper Class Reference}
\label{class_c_s_image_viewer_1_1pnm_helper}\index{CSImageViewer::pnmHelper@{CSImageViewer::pnmHelper}}
This class contains methods that read and write PNM/PGM/PPM images (color rgb and grey images).  


\subsection*{Static Public Member Functions}
\begin{CompactItemize}
\item 
static int[$\,$] {\bf read\_\-pnm\_\-file} (String fname, out int w, out int h, out int samples\-Per\-Pixel, out int min, out int max)
\begin{CompactList}\small\item\em {\bf  This method should be generally used to read any pnm (pgm grey, ppm color) binary or ascii image files. } \item\end{CompactList}\item 
static int[$\,$] {\bf read\_\-binary\_\-pgm16\_\-file} (String fname, out int w, out int h, out int min, out int max)
\begin{CompactList}\small\item\em {\bf Non-standard} function that reads 16-bit values as a binary pgm file ({\bf not implemented yet}). \item\end{CompactList}\item 
static int[$\,$] {\bf read\_\-binary\_\-pgm32\_\-file} (String fname, out int w, out int h, out int min, out int max)
\begin{CompactList}\small\item\em {\bf Non-standard} function that reads 32-bit values as a binary pgm file ({\bf not implemented yet}). \item\end{CompactList}\item 
static void {\bf write\_\-pgm\_\-or\_\-ppm\_\-ascii\_\-data} (String fname, int[$\,$] buff, int width, int height, int samples\_\-per\_\-pixel)
\begin{CompactList}\small\item\em Standard function that writes values as a pgm (grey) or ppm (color) ascii file. \item\end{CompactList}\item 
static void {\bf write\_\-binary\_\-pgm\_\-or\_\-ppm\_\-data8} (String fname, int[$\,$] buff, int width, int height, int samples\_\-per\_\-pixel)
\begin{CompactList}\small\item\em Standard functions that writes 8-bit values as a binary pgm (gray) or ppm (color) file. \item\end{CompactList}\item 
static void {\bf write\_\-binary\_\-pgm\_\-data16} (String fname, int[$\,$] buff, int width, int height)
\begin{CompactList}\small\item\em {\bf Non-standard} function that writes 16-bit values as a binary pgm (gray only) file. \item\end{CompactList}\item 
static void {\bf write\_\-binary\_\-pgm\_\-data32} (String fname, int[$\,$] buff, int width, int height)
\begin{CompactList}\small\item\em {\bf Non-standard} function that writes 32-bit values as a binary pgm (gray only) file. \item\end{CompactList}\end{CompactItemize}
\subsection*{Static Protected Member Functions}
\begin{CompactItemize}
\item 
static int[$\,$] {\bf read\_\-ascii\_\-pgm\_\-file} (String fname, out int w, out int h, out int min, out int max)
\begin{CompactList}\small\item\em This function reads an ascii gray pgm file. \item\end{CompactList}\item 
static int[$\,$] {\bf read\_\-ascii\_\-ppm\_\-file} (String fname, out int w, out int h, out int min, out int max)
\begin{CompactList}\small\item\em This function reads an ascii color ppm file. \item\end{CompactList}\item 
static int[$\,$] {\bf read\_\-binary\_\-pgm\_\-file} (String fname, out int w, out int h, out int min, out int max)
\begin{CompactList}\small\item\em This function reads a binary gray pgm file. \item\end{CompactList}\item 
static int[$\,$] {\bf read\_\-binary\_\-ppm\_\-file} (String fname, out int w, out int h, out int min, out int max)
\begin{CompactList}\small\item\em This function reads a binary color (rgb, 8-bits per component/24-bits per pixel) ppm file. \item\end{CompactList}\item 
static String {\bf read\-Header} (Stream\-Reader sr, out int w, out int h)
\begin{CompactList}\small\item\em Read image file header. \item\end{CompactList}\item 
static int[$\,$] {\bf read\_\-binary\_\-int8s} (String fname, int how\-Many, out int max, out int min)
\begin{CompactList}\small\item\em This function reads binary 8-bit integer values from a file. \item\end{CompactList}\item 
static int[$\,$] {\bf read\_\-binary\_\-int16s} (String fname, int how\-Many, out int max, out int min)
\begin{CompactList}\small\item\em This function reads binary 16-bit integer values from a file. \item\end{CompactList}\item 
static int[$\,$] {\bf read\_\-binary\_\-int32s} (String fname, int how\-Many, out int max, out int min)
\begin{CompactList}\small\item\em This function reads binary 32-bit integer values from a file. \item\end{CompactList}\item 
static int[$\,$] {\bf read\_\-ascii\_\-ints} (Stream\-Reader sr, int how\-Many, out int max, out int min)
\begin{CompactList}\small\item\em This function reads ascii integer (8-bits or more) values from a file. \item\end{CompactList}\end{CompactItemize}
\subsection*{Static Private Member Functions}
\begin{CompactItemize}
\item 
static void {\bf write\-String} (Binary\-Writer wr, String s)
\begin{CompactList}\small\item\em When I do binary writes of strings in C\#, is appears that they are stored as ascii counted strings (so the count appears before the string. This function simply writes out the characters in the string. \item\end{CompactList}\end{CompactItemize}


\subsection{Detailed Description}
This class contains methods that read and write PNM/PGM/PPM images (color rgb and grey images). 



\subsection{Member Function Documentation}
\index{CSImageViewer::pnmHelper@{CSImage\-Viewer::pnm\-Helper}!read_ascii_ints@{read\_\-ascii\_\-ints}}
\index{read_ascii_ints@{read\_\-ascii\_\-ints}!CSImageViewer::pnmHelper@{CSImage\-Viewer::pnm\-Helper}}
\subsubsection{\setlength{\rightskip}{0pt plus 5cm}static int [$\,$] read\_\-ascii\_\-ints (Stream\-Reader {\em sr}, int {\em how\-Many}, out int {\em max}, out int {\em min})\hspace{0.3cm}{\tt  [static, protected]}}\label{class_c_s_image_viewer_1_1pnm_helper_74b6ea40f46549d20fb016cf9aea875f}


This function reads ascii integer (8-bits or more) values from a file. 

\begin{Desc}
\item[Parameters:]
\begin{description}
\item[{\em sr}]is the input image file stream \item[{\em how\-Many}]is the number of values (not bytes) to read \item[{\em min}]will be set to min value in image \item[{\em max}]will be set to min value in image\end{description}
\end{Desc}
\begin{Desc}
\item[Returns:]array of pixel values as well as min and max will be set \end{Desc}
\index{CSImageViewer::pnmHelper@{CSImage\-Viewer::pnm\-Helper}!read_ascii_pgm_file@{read\_\-ascii\_\-pgm\_\-file}}
\index{read_ascii_pgm_file@{read\_\-ascii\_\-pgm\_\-file}!CSImageViewer::pnmHelper@{CSImage\-Viewer::pnm\-Helper}}
\subsubsection{\setlength{\rightskip}{0pt plus 5cm}static int [$\,$] read\_\-ascii\_\-pgm\_\-file (String {\em fname}, out int {\em w}, out int {\em h}, out int {\em min}, out int {\em max})\hspace{0.3cm}{\tt  [static, protected]}}\label{class_c_s_image_viewer_1_1pnm_helper_3b60f3edcaf62c8e0d8e01383f7e0e32}


This function reads an ascii gray pgm file. 

This type of file is formatted as follows: \small\begin{alltt}
    P2
    w h
    maxval
    v\_1 v\_2 v\_3 . . . v\_w*h
  \end{alltt}\normalsize 


\begin{Desc}
\item[Parameters:]
\begin{description}
\item[{\em fname}]input image file name \item[{\em w}]will be set to the image width \item[{\em h}]will be set to the image height \item[{\em min}]will be set to min value in image \item[{\em max}]will be set to min value in image\end{description}
\end{Desc}
\begin{Desc}
\item[Returns:]array of pixel values (gray) as well as w, h, min, and max will be set \end{Desc}
\index{CSImageViewer::pnmHelper@{CSImage\-Viewer::pnm\-Helper}!read_ascii_ppm_file@{read\_\-ascii\_\-ppm\_\-file}}
\index{read_ascii_ppm_file@{read\_\-ascii\_\-ppm\_\-file}!CSImageViewer::pnmHelper@{CSImage\-Viewer::pnm\-Helper}}
\subsubsection{\setlength{\rightskip}{0pt plus 5cm}static int [$\,$] read\_\-ascii\_\-ppm\_\-file (String {\em fname}, out int {\em w}, out int {\em h}, out int {\em min}, out int {\em max})\hspace{0.3cm}{\tt  [static, protected]}}\label{class_c_s_image_viewer_1_1pnm_helper_a184666e0242cb45d7604f8a9828db5f}


This function reads an ascii color ppm file. 

This type of file is formatted as follows: \small\begin{alltt}
    P3
    w h
    maxval
    vr\_1 vg\_1 vb\_1
    vr\_2 vg\_2 vb\_2
    . . .
    vr\_w*h vg\_w*h vb\_w*h
  \end{alltt}\normalsize 


\begin{Desc}
\item[Parameters:]
\begin{description}
\item[{\em fname}]input image file name \item[{\em w}]will be set to the image width \item[{\em h}]will be set to the image height \item[{\em min}]will be set to min value in image \item[{\em max}]will be set to min value in image\end{description}
\end{Desc}
\begin{Desc}
\item[Returns:]array of pixel values (color) as well as w, h, min, and max will be set \end{Desc}
\index{CSImageViewer::pnmHelper@{CSImage\-Viewer::pnm\-Helper}!read_binary_int16s@{read\_\-binary\_\-int16s}}
\index{read_binary_int16s@{read\_\-binary\_\-int16s}!CSImageViewer::pnmHelper@{CSImage\-Viewer::pnm\-Helper}}
\subsubsection{\setlength{\rightskip}{0pt plus 5cm}static int [$\,$] read\_\-binary\_\-int16s (String {\em fname}, int {\em how\-Many}, out int {\em max}, out int {\em min})\hspace{0.3cm}{\tt  [static, protected]}}\label{class_c_s_image_viewer_1_1pnm_helper_20a1395c35b68ef2a964d34b3474affb}


This function reads binary 16-bit integer values from a file. 

\begin{Desc}
\item[Parameters:]
\begin{description}
\item[{\em fname}]is the input image file name \item[{\em how\-Many}]is the the number of values (not bytes) to read \item[{\em min}]will be set to min value in image \item[{\em max}]will be set to min value in image\end{description}
\end{Desc}
\begin{Desc}
\item[Returns:]array of pixel values as well as min and max will be set \end{Desc}
\index{CSImageViewer::pnmHelper@{CSImage\-Viewer::pnm\-Helper}!read_binary_int32s@{read\_\-binary\_\-int32s}}
\index{read_binary_int32s@{read\_\-binary\_\-int32s}!CSImageViewer::pnmHelper@{CSImage\-Viewer::pnm\-Helper}}
\subsubsection{\setlength{\rightskip}{0pt plus 5cm}static int [$\,$] read\_\-binary\_\-int32s (String {\em fname}, int {\em how\-Many}, out int {\em max}, out int {\em min})\hspace{0.3cm}{\tt  [static, protected]}}\label{class_c_s_image_viewer_1_1pnm_helper_1cfac7ded3b4522ee9ee39776fe004a4}


This function reads binary 32-bit integer values from a file. 

\begin{Desc}
\item[Parameters:]
\begin{description}
\item[{\em fname}]is the input image file name \item[{\em how\-Many}]is the number of values (not bytes) to read \item[{\em min}]will be set to min value in image \item[{\em max}]will be set to min value in image\end{description}
\end{Desc}
\begin{Desc}
\item[Returns:]array of pixel values as well as min and max will be set \end{Desc}
\index{CSImageViewer::pnmHelper@{CSImage\-Viewer::pnm\-Helper}!read_binary_int8s@{read\_\-binary\_\-int8s}}
\index{read_binary_int8s@{read\_\-binary\_\-int8s}!CSImageViewer::pnmHelper@{CSImage\-Viewer::pnm\-Helper}}
\subsubsection{\setlength{\rightskip}{0pt plus 5cm}static int [$\,$] read\_\-binary\_\-int8s (String {\em fname}, int {\em how\-Many}, out int {\em max}, out int {\em min})\hspace{0.3cm}{\tt  [static, protected]}}\label{class_c_s_image_viewer_1_1pnm_helper_aec73e6bedb159d8b8f6b6258b4aff13}


This function reads binary 8-bit integer values from a file. 

\begin{Desc}
\item[Parameters:]
\begin{description}
\item[{\em fname}]is the input image file name \item[{\em how\-Many}]is the number of values (not bytes) to read \item[{\em min}]will be set to min value in image \item[{\em max}]will be set to min value in image\end{description}
\end{Desc}
\begin{Desc}
\item[Returns:]array of pixel values as well as min and max will be set \end{Desc}
\index{CSImageViewer::pnmHelper@{CSImage\-Viewer::pnm\-Helper}!read_binary_pgm16_file@{read\_\-binary\_\-pgm16\_\-file}}
\index{read_binary_pgm16_file@{read\_\-binary\_\-pgm16\_\-file}!CSImageViewer::pnmHelper@{CSImage\-Viewer::pnm\-Helper}}
\subsubsection{\setlength{\rightskip}{0pt plus 5cm}static int [$\,$] read\_\-binary\_\-pgm16\_\-file (String {\em fname}, out int {\em w}, out int {\em h}, out int {\em min}, out int {\em max})\hspace{0.3cm}{\tt  [static]}}\label{class_c_s_image_viewer_1_1pnm_helper_b5890bff21ce0535e4425a166e8dc7b6}


{\bf Non-standard} function that reads 16-bit values as a binary pgm file ({\bf not implemented yet}). 

\begin{Desc}
\item[Parameters:]
\begin{description}
\item[{\em fname}]input image file name \item[{\em w}]will be set to image width \item[{\em h}]will be set to image height \item[{\em min}]will be set to min value in image \item[{\em max}]will be set to min value in image\end{description}
\end{Desc}
\begin{Desc}
\item[Returns:]array of pixel values (gray) as well as w, h, min, and max will be set \end{Desc}
\index{CSImageViewer::pnmHelper@{CSImage\-Viewer::pnm\-Helper}!read_binary_pgm32_file@{read\_\-binary\_\-pgm32\_\-file}}
\index{read_binary_pgm32_file@{read\_\-binary\_\-pgm32\_\-file}!CSImageViewer::pnmHelper@{CSImage\-Viewer::pnm\-Helper}}
\subsubsection{\setlength{\rightskip}{0pt plus 5cm}static int [$\,$] read\_\-binary\_\-pgm32\_\-file (String {\em fname}, out int {\em w}, out int {\em h}, out int {\em min}, out int {\em max})\hspace{0.3cm}{\tt  [static]}}\label{class_c_s_image_viewer_1_1pnm_helper_5e15c7534f8d2fad9be751a869dc5efe}


{\bf Non-standard} function that reads 32-bit values as a binary pgm file ({\bf not implemented yet}). 

\begin{Desc}
\item[Parameters:]
\begin{description}
\item[{\em fname}]input image file name \item[{\em w}]will be set to image width \item[{\em h}]will be set to image height \item[{\em min}]will be set to min value in image \item[{\em max}]will be set to min value in image\end{description}
\end{Desc}
\begin{Desc}
\item[Returns:]array of pixel values (gray) as well as w, h, min, and max will be set \end{Desc}
\index{CSImageViewer::pnmHelper@{CSImage\-Viewer::pnm\-Helper}!read_binary_pgm_file@{read\_\-binary\_\-pgm\_\-file}}
\index{read_binary_pgm_file@{read\_\-binary\_\-pgm\_\-file}!CSImageViewer::pnmHelper@{CSImage\-Viewer::pnm\-Helper}}
\subsubsection{\setlength{\rightskip}{0pt plus 5cm}static int [$\,$] read\_\-binary\_\-pgm\_\-file (String {\em fname}, out int {\em w}, out int {\em h}, out int {\em min}, out int {\em max})\hspace{0.3cm}{\tt  [static, protected]}}\label{class_c_s_image_viewer_1_1pnm_helper_4998c73034df68afc294e0be9235e2cc}


This function reads a binary gray pgm file. 

This type of file is formatted as follows: \small\begin{alltt}
    P5
    w h
    maxval
    v\_1 v\_2 v\_3 . . . v\_w*h
  \end{alltt}\normalsize 
 One 8-bit byte per binary value.

\begin{Desc}
\item[Parameters:]
\begin{description}
\item[{\em fname}]input image file name \item[{\em w}]will be set to the image width \item[{\em h}]will be set to the image height \item[{\em min}]will be set to min value in image \item[{\em max}]will be set to min value in image\end{description}
\end{Desc}
\begin{Desc}
\item[Returns:]array of pixel values (gray) as well as w, h, min, and max will be set \end{Desc}
\index{CSImageViewer::pnmHelper@{CSImage\-Viewer::pnm\-Helper}!read_binary_ppm_file@{read\_\-binary\_\-ppm\_\-file}}
\index{read_binary_ppm_file@{read\_\-binary\_\-ppm\_\-file}!CSImageViewer::pnmHelper@{CSImage\-Viewer::pnm\-Helper}}
\subsubsection{\setlength{\rightskip}{0pt plus 5cm}static int [$\,$] read\_\-binary\_\-ppm\_\-file (String {\em fname}, out int {\em w}, out int {\em h}, out int {\em min}, out int {\em max})\hspace{0.3cm}{\tt  [static, protected]}}\label{class_c_s_image_viewer_1_1pnm_helper_995555dda732256678609a16427524aa}


This function reads a binary color (rgb, 8-bits per component/24-bits per pixel) ppm file. 

This type of file is formatted as follows: \small\begin{alltt}
    P6
    w h
    maxval
    vr\_1 vg\_1 vb\_1 
    vr\_2 vg\_2 vb\_2
    . . .
    vr\_w*h vg\_w*h vb\_w*h
  \end{alltt}\normalsize 
 One 8-bit byte per binary value (3 bytes for each RGB value).

\begin{Desc}
\item[Parameters:]
\begin{description}
\item[{\em fname}]input image file name \item[{\em w}]will be set to the image width \item[{\em h}]will be set to the image height \item[{\em min}]will be set to min value in image \item[{\em max}]will be set to min value in image\end{description}
\end{Desc}
\begin{Desc}
\item[Returns:]array of pixel values (color) as well as w, h, min, and max will be set \end{Desc}
\index{CSImageViewer::pnmHelper@{CSImage\-Viewer::pnm\-Helper}!read_pnm_file@{read\_\-pnm\_\-file}}
\index{read_pnm_file@{read\_\-pnm\_\-file}!CSImageViewer::pnmHelper@{CSImage\-Viewer::pnm\-Helper}}
\subsubsection{\setlength{\rightskip}{0pt plus 5cm}static int [$\,$] read\_\-pnm\_\-file (String {\em fname}, out int {\em w}, out int {\em h}, out int {\em samples\-Per\-Pixel}, out int {\em min}, out int {\em max})\hspace{0.3cm}{\tt  [static]}}\label{class_c_s_image_viewer_1_1pnm_helper_5f89692349ec01bcc8017090c900c15f}


{\bf  This method should be generally used to read any pnm (pgm grey, ppm color) binary or ascii image files. } 

\begin{Desc}
\item[Returns:]an int array of pixel values; in the case of gray data, each entry is an array value; in the case of color data, the first entry is the red, the second the green, and the third the blue.\end{Desc}
\begin{Desc}
\item[Parameters:]
\begin{description}
\item[{\em fname}]input image file name \item[{\em w}]will be set to image width \item[{\em h}]will be set to image height \item[{\em samples\-Per\-Pixel}]samples\-Per\-Pixel[0] will be set to 1 (gray) or 3 (color) \item[{\em min}]will be set to min value in image \item[{\em max}]will be set to min value in image\end{description}
\end{Desc}
\begin{Desc}
\item[Returns:]array of pixel values (rgb or gray) as well as w, h, samples\-Per\-Pixel, min, and max will be set \end{Desc}
\index{CSImageViewer::pnmHelper@{CSImage\-Viewer::pnm\-Helper}!readHeader@{readHeader}}
\index{readHeader@{readHeader}!CSImageViewer::pnmHelper@{CSImage\-Viewer::pnm\-Helper}}
\subsubsection{\setlength{\rightskip}{0pt plus 5cm}static String read\-Header (Stream\-Reader {\em sr}, out int {\em w}, out int {\em h})\hspace{0.3cm}{\tt  [static, protected]}}\label{class_c_s_image_viewer_1_1pnm_helper_a99b654bc67e0f3e658ea79b92aa7b3c}


Read image file header. 

\begin{Desc}
\item[Parameters:]
\begin{description}
\item[{\em sr}]is the input image stream \item[{\em w}]will be set to image width \item[{\em h}]will be set to image height\end{description}
\end{Desc}
\begin{Desc}
\item[Returns:]string indicating image file type as well as w and h will be set \end{Desc}
\index{CSImageViewer::pnmHelper@{CSImage\-Viewer::pnm\-Helper}!write_binary_pgm_data16@{write\_\-binary\_\-pgm\_\-data16}}
\index{write_binary_pgm_data16@{write\_\-binary\_\-pgm\_\-data16}!CSImageViewer::pnmHelper@{CSImage\-Viewer::pnm\-Helper}}
\subsubsection{\setlength{\rightskip}{0pt plus 5cm}static void write\_\-binary\_\-pgm\_\-data16 (String {\em fname}, int[$\,$] {\em buff}, int {\em width}, int {\em height})\hspace{0.3cm}{\tt  [static]}}\label{class_c_s_image_viewer_1_1pnm_helper_b9389279b59ecaff8492133a68a59b98}


{\bf Non-standard} function that writes 16-bit values as a binary pgm (gray only) file. 

\begin{Desc}
\item[Parameters:]
\begin{description}
\item[{\em fname}]input image file name \item[{\em buff}]buffer of pixel values to write to the file \item[{\em width}]image width \item[{\em height}]image height\end{description}
\end{Desc}
\begin{Desc}
\item[Returns:]nothing (void)\end{Desc}
\begin{Desc}
\item[{\bf Todo}]below could be made more efficient by replacing it with fewer, larger buffer writes \end{Desc}
\index{CSImageViewer::pnmHelper@{CSImage\-Viewer::pnm\-Helper}!write_binary_pgm_data32@{write\_\-binary\_\-pgm\_\-data32}}
\index{write_binary_pgm_data32@{write\_\-binary\_\-pgm\_\-data32}!CSImageViewer::pnmHelper@{CSImage\-Viewer::pnm\-Helper}}
\subsubsection{\setlength{\rightskip}{0pt plus 5cm}static void write\_\-binary\_\-pgm\_\-data32 (String {\em fname}, int[$\,$] {\em buff}, int {\em width}, int {\em height})\hspace{0.3cm}{\tt  [static]}}\label{class_c_s_image_viewer_1_1pnm_helper_37c7b83d8f17d8e3c8a8ecafe511d070}


{\bf Non-standard} function that writes 32-bit values as a binary pgm (gray only) file. 

\begin{Desc}
\item[Parameters:]
\begin{description}
\item[{\em fname}]input image file name \item[{\em buff}]buffer of pixel values to write to the file \item[{\em width}]image width \item[{\em height}]image height\end{description}
\end{Desc}
\begin{Desc}
\item[Returns:]nothing (void)\end{Desc}
\begin{Desc}
\item[{\bf Todo}]below could be made more efficient by replacing it with fewer, larger buffer writes \end{Desc}
\index{CSImageViewer::pnmHelper@{CSImage\-Viewer::pnm\-Helper}!write_binary_pgm_or_ppm_data8@{write\_\-binary\_\-pgm\_\-or\_\-ppm\_\-data8}}
\index{write_binary_pgm_or_ppm_data8@{write\_\-binary\_\-pgm\_\-or\_\-ppm\_\-data8}!CSImageViewer::pnmHelper@{CSImage\-Viewer::pnm\-Helper}}
\subsubsection{\setlength{\rightskip}{0pt plus 5cm}static void write\_\-binary\_\-pgm\_\-or\_\-ppm\_\-data8 (String {\em fname}, int[$\,$] {\em buff}, int {\em width}, int {\em height}, int {\em samples\_\-per\_\-pixel})\hspace{0.3cm}{\tt  [static]}}\label{class_c_s_image_viewer_1_1pnm_helper_60a4449366845ae713e39edda7adfed1}


Standard functions that writes 8-bit values as a binary pgm (gray) or ppm (color) file. 

\begin{Desc}
\item[Parameters:]
\begin{description}
\item[{\em fname}]input image file name \item[{\em buff}]buffer of pixel values to write to the file \item[{\em width}]image width \item[{\em height}]image height \item[{\em samples\_\-per\_\-pixel}]samples per pixel (1=gray; 3=rgb/color)\end{description}
\end{Desc}
\begin{Desc}
\item[Returns:]nothing (void)\end{Desc}
\begin{Desc}
\item[{\bf Todo}]below could be made more efficient by replacing it with fewer, larger buffer writes \end{Desc}
\index{CSImageViewer::pnmHelper@{CSImage\-Viewer::pnm\-Helper}!write_pgm_or_ppm_ascii_data@{write\_\-pgm\_\-or\_\-ppm\_\-ascii\_\-data}}
\index{write_pgm_or_ppm_ascii_data@{write\_\-pgm\_\-or\_\-ppm\_\-ascii\_\-data}!CSImageViewer::pnmHelper@{CSImage\-Viewer::pnm\-Helper}}
\subsubsection{\setlength{\rightskip}{0pt plus 5cm}static void write\_\-pgm\_\-or\_\-ppm\_\-ascii\_\-data (String {\em fname}, int[$\,$] {\em buff}, int {\em width}, int {\em height}, int {\em samples\_\-per\_\-pixel})\hspace{0.3cm}{\tt  [static]}}\label{class_c_s_image_viewer_1_1pnm_helper_3c2224c0966249ea2ff6c0981de771ee}


Standard function that writes values as a pgm (grey) or ppm (color) ascii file. 

\begin{Desc}
\item[Parameters:]
\begin{description}
\item[{\em fname}]input image file name \item[{\em buff}]buffer of pixel values to write to the file \item[{\em width}]image width \item[{\em height}]image height \item[{\em samples\_\-per\_\-pixel}]samples per pixel (1=gray; 3=rgb/color)\end{description}
\end{Desc}
\begin{Desc}
\item[Returns:]nothing (void) \end{Desc}
\index{CSImageViewer::pnmHelper@{CSImage\-Viewer::pnm\-Helper}!writeString@{writeString}}
\index{writeString@{writeString}!CSImageViewer::pnmHelper@{CSImage\-Viewer::pnm\-Helper}}
\subsubsection{\setlength{\rightskip}{0pt plus 5cm}static void write\-String (Binary\-Writer {\em wr}, String {\em s})\hspace{0.3cm}{\tt  [static, private]}}\label{class_c_s_image_viewer_1_1pnm_helper_c2861a988c1bc2b0fabc9d21f0e71167}


When I do binary writes of strings in C\#, is appears that they are stored as ascii counted strings (so the count appears before the string. This function simply writes out the characters in the string. 

\begin{Desc}
\item[Parameters:]
\begin{description}
\item[{\em wr}]output binary stream \item[{\em s}]string to write\end{description}
\end{Desc}
\begin{Desc}
\item[Returns:]nothing (void) \end{Desc}


The documentation for this class was generated from the following file:\begin{CompactItemize}
\item 
{\bf pnm\-Helper.cs}\end{CompactItemize}
